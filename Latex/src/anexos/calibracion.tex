\chapter{Calibración de los sensores.}

\todo[inline]{Enseñar los errores que pueden dar las capturas y explicar que es un problema de calibración y como se podría mejorar.}

% Por otro lado, buscando investigaciones relacionadas con el sensor, en ``Análisis comparativo de métodos de calibrado para sensores RGB-D y su influencia en el registro de múltiples vistas'' de \citep{VillenaMartinez2015} se demuestra que los parámetros por defecto que se utilizan en los sensores \gls{rgbd} pueden no ser lo suficientemente precisos.
% Por tanto es necesario hacer un estudio previo de métodos de calibrado para la obtención de mejores resultados.
% El calibrado en sensores \gls{rgbd} consiste en encontrar los parámetros adecuados para el sensor permitiendo mejorar la exactitud de los datos obtenidos.
% Estos datos consistirán en la información del color y la profundidad en la escena.

% Existen dos tipos de calibrados:
% intrínseco y extrínseco.

% \begin{description}
%     \item[Calibrado intrínseco:]
    
%     es aquel que se centra en los parámetros internos del sensor.

%     Uno de esos parámetros es la distancia entre el centro óptico y el plano focal donde se proyecta la imagen, conocida como distancia o longitud focal.
%     Ésta permite estimar el campo de visión de la cámara.

%     Otro parámetro es el punto principal que representa el desplazamiento del eje óptico respecto al sensor, provocando un desplazamiento en la proyección de la imagen.

%     Las distorsiones ópticas debido a fallos en el diseño o fabricación de las lentes es otro parámetro a tener en cuenta.
%     Esta distorsión puede ser una distorsión radial (lentes que no son perfectamente parabólicas) o tangencial (lente que no es paralela totalmente con el sensor).

%     \item[Calibrado extrínseco:]
    
%     se refiere al tipo de calibrado que se centra en las características externas.
%     Un ejemplo de ello sería el alineamiento de una cámara estereoscópica.

%     Consiste en el proceso de situar las imágenes obtenidas por ambos sensores del mismo sistema.
%     Para ello es necesario conocer las matrices de rotación y traslación que permitan aplicar la transformación necesaria para que las imágenes obtenidas queden situadas en un eje de coordenadas común.
% \end{description}

% En ``A Quantitative Comparison of Calibration Methods for RGB-D Sensors Using Different Technologies'', trabajo de investigación de \citep{VillenaMartinez2017} se trató de analizar tres métodos distintos de calibración con la finalidad de encontrar el mejor método de calibrado para mejorar la exactitud de los datos obtenidos.
% Se comparan los métodos de Bouguet, Burrus y Herrera.

% El método de \citep{bouguet2004camera} intenta calibrar la cámara RGB y el sensor de profundidad como si se tratase de un sistema estéreo, aplicando transformaciones para obtener las correspondencias entre el mapa de profundidad y la imagen de color.

% El método de \citep{burrus2012rgbdemo} también realiza el proceso de calibración como si se tratase de un sistema estéreo.
% A partir de la información del patrón en ambas cámaras se realiza una calibración para obtener los parámetros extrínsecos que permiten alinear las imágenes de ambas c    ámaras (color y profundidad).

% El método de \citep{HerreraC.2012} permite calibrar simultáneamente la posición relativa entre dos cámaras de color y una cámara de profundidad.
% Además, ha sido diseñado con el objetivo de ser aplicable a varios sensores.

% De todas las pruebas realizadas en dicha investigación, el método de Herrera es el que mejores resultados obtiene generalmente.

% También merece la pena tener en cuenta un algoritmo llamado \gls{mclse}.
% En el artículo científico ``Iterative multilinear optimization for planar model fitting under geometric constraints'' de \citep{Azorin-Lopez2021} presentan este algoritmo como un método de calibrado extrínseco para el cálculo de modelos planares con restricciones.
% Se trata de un algoritmo muy rápido y preciso.