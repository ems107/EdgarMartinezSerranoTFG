%%%%%%%%%%%%%%%%%%%%%%%%%%%%%%%%%%%%%%%%%%%%%%%%%%%%%%%%%%%%%%%%%%%%%%%%
% Plantilla TFG/TFM
% Escuela Politécnica Superior de la Universidad de Alicante
% Realizado por: Jose Manuel Requena Plens
% Contacto: info@jmrplens.com / Telegram:@jmrplens
%%%%%%%%%%%%%%%%%%%%%%%%%%%%%%%%%%%%%%%%%%%%%%%%%%%%%%%%%%%%%%%%%%%%%%%%

% \cleardoublepage %salta a nueva página impar
\chapter*{Preámbulo}

\thispagestyle{empty}
\vspace{1cm}

Este \gls{tfg} consiste en desarrollar un sistema embebido de bajo coste y portable que, utilizando datos \gls{rgbd} de color y profundidad, reconstruya un cuerpo humano completo.
El sistema tomará capturas del cuerpo desde un punto de vista fijo mientras el sujeto gira frente a la cámara.
Con los datos obtenidos, se aplicarán técnicas de registro para alinear las distintas vistas y así formar el modelo completo del sujeto.
Estos métodos, se acelerarán posteriormente con la \gls{gpu} del sistema embebido para permitir una ejecución en línea del sistema.

\cleardoublepage %salta a nueva página impar


